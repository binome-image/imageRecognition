\documentclass{report}

\title{Rapport Projet Image}
\author{Patrice Coudert}
\usepackage{xcolor}
\usepackage[francais]{babel}
\usepackage[T1]{fontenc}
\usepackage[utf8]{inputenc}
\usepackage{tikz}
\usepackage[nottoc, notlof, notlot]{tocbibind}
\usepackage{amsthm}
\usepackage{stmaryrd} 
\usepackage{algorithm}
\usepackage{algorithmic}
\usepackage{lmodern}
\usepackage{amsmath,amsfonts}
\newcommand{\TextSoulign}[3]{\emph{\textcolor{#1}{\underline{#2\textcolor{black}{#3}}}}}
\newtheorem{lemme}{Lemme}
\newtheorem{theoreme}{Théorème}    
\newtheorem{mydef}{Définition}
\newtheorem{prop}{Propriété}    
\newtheorem{res}{Résultat}     
\usepackage{tkz-graph}
\usetikzlibrary{trees,decorations,decorations.pathmorphing,decorations.pathreplacing}                                      
\begin{document}

\begin{titlepage}
	\begin{center}
		\vspace{3.5cm}
		\huge{Rapport du Projet d'image}
		\vspace{1cm}
		
		\Huge{\TextSoulign{red}{}{Reconnaissance et indexation}}

		\Huge{\TextSoulign{red}{}{de forme}}
		\vspace{5cm}
		
		\LARGE{Patrice Coudert et Frédéric Lang}
		\vspace{3.5cm}
		
	\end{center}
	
	
	
\end{titlepage}

\tableofcontents

\chapter*{Introduction et structure du rapport}

Le principe du projet est de créer une signature. Pour cela nous avons créé un certains nombre d'indicateurs
que nous allons tout d'abord vous présenter. On en profitera pour s'interesser à leur robustesse vis à vis de l'occlusion, du bruit, ...
Ces propriétés seront étudiés à la fois théoriquement et en pratique.

Pour les résultats pratiques, les scripts bash et les graphiques sont présent dans le dossier pour reporduire ces tests. Même si les graphiques
ne seront pas tous dans la rapport, il sont présent dans le dossier résultat. Les scripts bash servent à reproduire les test que nous 
avons effectués.

La suite expliquera comment nous utilisons ces indicateurs pour répondre aux questions. Nous proposons deux options: la première renverra la classe
la plus probable et la deuxième renverra les k images les plus proches.

\section*{Explications du programme}

Nous allons commencer expliquer comment utiliser notre programme. Le programme s'appelle $main$. Il prend deux options. La première est
le mode du programme. Les différents options sont:

\noindent-$-help$: qui affiche les options et les explications de ce qu'elles font.
\noindent-$-class$: qui renvoie la classe la plus probable et la probabilité que l'image soit dans cette classe.
\noindent-$-simil n$: qui renvoie les $n$ images les plus proches au sens de notre distance.
\noindent-$-indn$: qui renvoie l'indicateur $n$ pour l'image en entrée.
\noindent-$-disp$: qui crée un fichier $image.eps$ qui 


\section{Détails des indicateurs}
Nous avons choisis de construire nos indicateurs à partir d'éléments géométriques qu'on peut extraire de l'image. De cette manière, à partir de la sortie graphique du programme, la cohérence des indicateurs peut-être vérifiée visuellement, ou une valeur non concordante pour une image donnée avec le reste de sa classe peut être expliquée par une défaillance dans l'implémentation des algorithmes. Pour que nos indicateurs soient invariants par transformation linéaire, à chaque fois nous calculons un rapport avec la racine carrée de la surface de l'image, qui correspondrait à l'échelle de l'image. Ci-dessous sont présenté nos indicateurs :

\begin{enumerate}
\item PERIMETRE : Cet indicateur est lié à la longueur du contour de l'image. Il est destiné à différencier les formes très courbés de celle qui se rapproche plus d'un disque, et de ce fait on un rapport périmètre/surface assez faible. Il est calculé en extrayant le contour de la forme avec la fonction \textbf{contour} inplémentée dans le fichier \textbf{utils.cpp}. Le contour obtenu correspond a une topologie 4-8.
\item DISQUECIRCONS : A Partir de la position du barycentre et de la distance du point le plus éloigné de celui-ci, calculé par les fonctions \textbf{barycentre} et \textbf{distFarthestPoint} dans \textbf{utils.cpp}, on calcule le rapport entre la surface du disque circonscrit à la forme et la surface de de la forme.
\item NBSEG : Le nombre de segment qui délimitent l'enveloppe convexe de l'image. Les points qui relient ces segments sont calculé par l'algorithme de Graham inplémenté par la fonction \textbf{convexHull} dans \textbf{convexhull.cpp}. Pour la première étape de l'algorithme où sont triés les points selon un ordre lié à un point \textbf{p0}, le tri n'a d'abord pas fonctionné car des points étaient alignés avec \textbf{p0}, mais ce problème a été corrigé en prenant en compte la distance avec celui-ci.
% todo : verifier algo
\item MAXSEG : Cet indicateur mesure la taille du plus grand segment délimitant l'enveloppe convexe.
\end{enumerate}

\end{document}

