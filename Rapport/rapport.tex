\documentclass{report}

\title{Rapport Stage L3}
\author{Patrice Coudert}
\usepackage{xcolor}
\usepackage[francais]{babel}
\usepackage[T1]{fontenc}
\usepackage[utf8]{inputenc}
\usepackage{tikz}
\usepackage[nottoc, notlof, notlot]{tocbibind}
\usepackage{amsthm}
\usepackage{stmaryrd} 
\usepackage{algorithm}
\usepackage{algorithmic}
\usepackage{lmodern}
\usepackage{amsmath,amsfonts}
\newcommand{\TextSoulign}[3]{\emph{\textcolor{#1}{\underline{#2\textcolor{black}{#3}}}}}
\newtheorem{lemme}{Lemme}
\newtheorem{theoreme}{Théorème}    
\newtheorem{mydef}{Définition}
\newtheorem{prop}{Propriété}    
\newtheorem{res}{Résultat}     
\usepackage{tkz-graph}
\usetikzlibrary{trees,decorations,decorations.pathmorphing,decorations.pathreplacing}                                      
\begin{document}

\begin{titlepage}
	\begin{center}
		\vspace{3.5cm}
		\huge{Rapport du Projet d'image}
		\vspace{1cm}
		
		\Huge{\TextSoulign{red}{}{Reconnaissance et indexation}}

		\Huge{\TextSoulign{red}{}{de forme}}
		\vspace{5cm}
		
		\LARGE{Patrice Coudert et Frédéric Lang}
		\vspace{3.5cm}
		
	\end{center}
	
	
	
\end{titlepage}

\tableofcontents

\chapter*{Introduction et structure du rapport}

Le principe du projet est de créer une signature. Pour cela nous avons créé un certains nombre d'indicateurs
que nous allons tout d'abord vous présenter. On en profitera pour s'interesser à leur robustesse vis à vis de l'occlusion, du bruit, ...
Ces propriétés seront étudiés à la fois théoriquement et en pratique.

Pour les résultats pratiques, les scripts bash et les graphiques sont présent dans le dossier pour reporduire ces tests. Même si les graphiques
ne seront pas tous dans la rapport, il sont présent dans le dossier résultat. Les scripts bash servent à reproduire les test que nous 
avons effectués.

La suite expliquera comment nous utilisons ces indicateurs pour répondre aux questions. Nous proposons deux options: la première renverra la classe
la plus probable et la deuxième renverra les k images les plus proches.

\section*{Explications du programme}

Nous allons commencer expliquer comment utiliser notre programme. Le programme s'appelle $main$. Il prend deux options. La première est
le mode du programme. Les différents options sont:

\noindent-$-help$: qui affiche les options et les explications de ce qu'elles font.
\noindent-$-class$: qui renvoie la classe la plus probable et la probabilité que l'image soit dans cette classe.
\noindent-$-simil n$: qui renvoie les $n$ images les plus proches au sens de notre distance.
\noindent-$-indn$: qui renvoie l'indicateur $n$ pour l'image en entrée.
\noindent-$-disp$: qui crée un fichier $image.eps$ qui 


\end{document}

